\documentclass[10pt]{scrartcl}

\usepackage[utf8]{inputenc}
\usepackage[english]{babel}

\usepackage[backend=biber]{biblatex}
\addbibresource{references.bib}

\usepackage{alltt}

\usepackage{holtexbasic}
\usepackage{holtex}

\usepackage{environ}
\NewEnviron{holthmenv}{%
  \[
    \scalebox{1.0}{\normalsize\rmfamily\color[rgb]{0,0,0}
      \begin{minipage}[c]{\textwidth}
      \begin{array}[t]{l}
        \BODY
      \end{array}
    \end{minipage}}
  \]}

\usepackage{listings}

\usepackage{hyperref}
\usepackage{todonotes}
\newcommand{\heiko}[1]{\todo[inline,author=Heiko,bordercolor=red!20,color=orange!20]{#1}}
\newcommand{\ekey}[1]{\texttt{#1}}

\title{A Programmer's Guide to Proving Theorems with HOL4}
\author{Anonymous}
\date{}

\begin{document}
\maketitle{}

\paragraph*{Summary}
Interactive theorem proving is a method to perform trustworthy, rigorous
mathematical proofs checked by computers.
From a programmer's point of view, learning an ITP system is similar to
learning a new programming language.
The main difference being that an ITP system is not only used to implement
a particular system, but also to prove theorems about it.
This guide is written for experienced programmers, that have done so-called
pen-and-paper proofs before, but never worked with any ITP system.
If you ever wanted to replace that nice, cozy thought of \emph{this function
should work as I intend it to} with the high-assurance of a rigorous
mathematical proof, this guide is for you\footnote{If you are asking yourself the question how this is even possible, the same reasoning applies.}.

\section{Introduction/Setup}
%
As with any programming language, picking up an ITP system first requires
installing the system.
In this guide we use the HOL4~\cite{HOL4web} theorem prover developed by ....
The guide also uses a tool called \emph{Lassie} which makes learning HOL4 easier
for newcomers.

Thus, following this guide requires installations of \texttt{polyML v5.7.1},
\texttt{Apache Ant v.1.9.9}, and \texttt{ruby 2.3.3}.
We provide a setup script that makes sure all the dependencies are installed,
and downloads and installs the latest versions of HOL4 and Lassie with this
guide (\url{https://todo.mpi-sws.org}).
Running the setup script will also set the variables \texttt{HOLDIR}, and
\texttt{LASSIEDIR} once, but they must always be set to the directory to which
HOL4, respectively Lassie was cloned, before running HOL4.

\subsection{Running HOL4}

We will explaining using HOL4 using the emacs editor.
There exist also plugins for vim, and Sublime text.

To interact with HOL4 within emacs, it suffices to append the following lines to
the \texttt{.emacs} file:
\begin{lstlisting}
(load-file "<HOLDIR>/tools/hol-unicode.el")
(load-file "<HOLDIR>/tools/hol-input.el")
(load-file "<HOLDIR>/tools/holscript-mode.el")
(load-file "<HOLDIR>/tools/hol-mode.el")
\end{lstlisting}

In general, HOL4 files always end with the suffix \verb!Script.sml!.
For example, a theory about natural numbers should be called
\verb!naturalNumbersScript.sml!.
In HOL4 speak, these files are called \emph{script files}.

After opening a script file in emacs, HOL4 is started by presssing
\ekey{M-h H}.
emacs will then interactively prompt for a position of the HOL4
read-eval-print loop (REPL).

\section{Interacting with the HOL4 REPL}

After starting the HOL4 REPL, code is send via \ekey{M-h r}.
The HOL4 REPL is an extended version of the polyML REPL, and as such can be like
the REPL's of other interpreted languages like e.g. ruby and python.

For example, running
\begin{lstlisting}
  3 + 5;
\end{lstlisting}

should result in the following output:
\begin{lstlisting}
> 3+5;
val it = 8: int
\end{lstlisting}

All of the functionality of the polyML REPL, and in general, the Standard ML
basis library (see e.g. \url{https://smlfamily.github.io/Basis/} for a reference)
are available in the HOL4 REPL.
Thus allowing simple creation, manipulation of lists, strings, options, and
simple I/O.

\section{A First (Pen-and-Paper Style) Proof}
%
The HOL4 features presented are exactly the same as for any other interpreted
programming language.
Next, we will turn to defining our first function in HOL4, and proving a first
theorem.

Take function \verb!sum n!, summing up the numbers from $0$ to $n$.
In HOL4 this function is defined by the code:
\begin{lstlisting}
Definition sum_def:
  sum n = if (n = 0) then 0 else n + sum (n-1)
End
\end{lstlisting}

Alternatively, HOL4 also supports defining a function by a system of equations,
moving the \texttt{case} expression of the \verb!sum! function to the outside:
\begin{lstlisting}
Definition sum_def:
  sum 0 = 0 /\
  sum n = n + sum (n-1)
End
\end{lstlisting}

Choosing one over the other has different benefits.
As a rule of thumb, it is recommended to prefer the system of equations if
the function could be implemented with a top-level \texttt{case} expression.

As a simple, first example, we will prove the closed form for summing $n$ numbers, also known as the gaussian sum:
\[
  \sum_{i=0}^{n} i = \frac{n * (n + 1)}{2}
\]

In HOL4 this goal this theorem is stated as
\begin{alltt}
Theorem gaussian_sum:
  ! n. sum n = n * (n + 1) DIV 2
Proof
QED
\end{alltt}

Marking the center line and running \ekey{M-h g}
we start an interactive HOL4 proof, and the REPL shows:

\begin{lstlisting}[mathescape=true]
> val it =
   Proof manager status: 1 proof.
   1. Incomplete goalstack:
        Initial goal:
        $\forall$n. sum n = n * (n + 1) DIV 2
   : proofs
\end{lstlisting}

In HOL4, theorems are proven by applying so-called \emph{tactics} to the goal
until the proof is finished.
To make it easier to pick up HOL4, we perform the first proof with Lassie, which
allows us to prove the theorem using natural language.

First, we load the Lassie library by running \lstinline{open LassieLib;} and
loading the jargon for performing mathematical proofs
(\lstinline{val _ = LassieLib.loadJargon "Arithmetic";}).

Following the intuition of a rigorous mathematical proof, we start a proof by
induction:
\begin{lstlisting}
nltac `Induction on 'n'.`
\end{lstlisting}

Here, \lstinline{nltac} is Lassie's function for parsing natural language and translating it into a HOL4 tactic.
The parameter \lstinline{`Induction on 'n'.`} is the actual tactic being applied.
To apply the tactic, the text must be marked and run with \ekey{M-h e}.
After running the code, the HOL4 REPL shows
\begin{lstlisting}
TODO
\end{lstlisting}

As for a pen-and-paper proof, the base case of the induction is trivial, and
solved with the tactic \lstinline{nltac `simplify.`}, leaving us only with the
second goal from above.

As for a pen-and-paper proof, the first step on the inductive case is to
simplify:
\begin{lstlisting}
nltac `simplify with [GSYM ADD_DIV_ADD_DIV, GSYM DIV2_def].`
\end{lstlisting}

Here, \lstinline{ADD_DIV_ADD_DIV} and \lstinline{DIV2_def} are theorems that are
used to enrich the simplifier with the additional theorems.
To find out their statements, mark the theorem names only, and send them to the
REPL with \ekey{M-h r}.
Sending \lstinline{GSYM} to the REPL shows that the function has type
\lstinline{:thm -> thm}, meaning that it takes a theorem as input and returns a
theorem.
Sending \lstinline{GSYM DIV2_def} and \lstinline{DIV2_def} to the REPL, one can
quickly see that \lstinline{GSYM} flips equalities in theorems.
This is sometimes useful as rewriting in HOL4 is directed from left-to-right.
If we have a theorem showing $f x = b$, HOL4 will simplify any occurence of
$f x$ into an occurence of $b$, but it will never replace occurences of $b$
with occurences of $f x$.

%   it suffices to show that the arguments are equal.
%   show 'SUC n * (SUC n + 1) = (SUC n + 1) + n * (SUC n + 1)' using (simplify with [MULT_CLAUSES]).
%   simplify.
%   show 'n * (n + 1) = SUC n * n' using (trivial using [MULT_CLAUSES, MULT_SYM]).
%   simplify.’

\section{References}
\printbibliography

\end{document}